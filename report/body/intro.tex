\section{个人课题研究方向}
% 研究方向概述
我参与的研究方向主要是探索AI在电子设计自动化(EDA)中的应用,
具体包括使用图神经网络(Graph Neural Network, GNN)完成芯片版图热点检测;
以及使用强化学习(Reinforcement Learning, RL)方法,完成光学邻近效应校正(Optical Proximity Correction, OPC)任务。

% 研究背景
随着集成电路设计规模不断增大,传统EDA工具在布局布线、时序优化和设计验证等环节的效率逐渐受到限制。
人工智能(AI)技术在模式识别、优化搜索和复杂数据分析方面具有优势,为EDA流程提供了新的探索方向。
AI在EDA中的应用,即"AI for EDA",可以提高设计流程的效率。


% 基于GNN的热点检测
热点检测旨在在版图层级识别难以光刻的几何模式,即热点(hotspots),以便在布局布线阶段提前定位并修正,提高芯片良率。
随着工艺缩小与工艺窗口收敛,热点更隐蔽;规则/模板匹配在复杂大规模设计上往往不足。
我们将版图布局表示为图结构,
并参与使用图神经网络学习布局特征,实现热点区域识别。


% 基于RL的OPC优化
光刻过程中,衍射效应导致在晶圆上生产的图形与设计版图会有偏差。
OPC通过调整掩模图形来补偿这种偏差,以确保最终图形符合设计要求。
传统OPC方法依赖物理模型和迭代修正,计算开销大。
我们搭建了深度强化学习模型,同样将版图建模为另一种图结构,使用循环神经网络处理序列化的图节点特征,
并通过强化学习方法学习优化掩模图形的策略,从而提高OPC效率。


目前,AI在EDA中的应用更多处于探索阶段,仍面临数据获取、模型泛化和集成到现有EDA流程等挑战。
我自己也刚刚开始接触相关研究工作,希望通过不断学习和实践,深入理解AI技术在EDA中的潜力与局限,
为未来芯片设计流程的智能化发展贡献力量。