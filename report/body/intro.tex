\section{引言}

在本次课程实验中,我们小组选择围绕《吃豆人》游戏环境,探索和实现了多种强化学习算法。
其中我负责训练环境的搭建,并实现了蒙特卡洛学习(Monte Carlo Learning)、Q-Learning 以及 Approximate Q-Learning 三种基于值函数的强化学习方法。

\subsection{Pacman 游戏}
% 简述
在《吃豆人》游戏中,玩家控制吃豆人(Pacman)在迷宫中上下左右移动,
地图中存在固定数量的食物(food)和胶囊(capsule),以及随机移动的鬼(ghost)。
当吃豆人吃掉所有食物时,游戏胜利。

% 规则
游戏的基础规则可以简单概括为:
吃豆人吃掉所有食物,游戏胜利;
吃豆人碰到鬼,游戏失败。
此外,
当吃豆人吃掉胶囊后,一定时间内所有鬼会进入恐慌(scared)状态;
此时吃豆人碰到鬼不会失败,而是变为吃掉鬼得分;
而被吃掉的鬼会在地图中的固定位置复活,并解除恐慌。

% 吃豆人作为强化学习环境

《吃豆人》作为一个经典游戏,%规则简单但需要复杂的决策,
% 适合用来进行强化学习算法的训练与测试。
% 该环境
% 具有以下特点:
% \begin{itemize}
%     \item 状态空间大小适中,奖励结构清晰,既不过于简单也不过于复杂,适合算法的实现与测试;
%     \item 环境兼具确定性与随机性,鬼的随机移动增加了不确定性,考验算法的鲁棒性;
%     \item 具有长期规划特性,agent 需要在避开鬼和收集食物之间进行权衡,胶囊机制增加了策略的多样性。
% \end{itemize}
其状态空间大小适中,奖励结构清晰,既不过于简单也不过于复杂,适合算法的实现与测试;
环境兼具确定性与随机性,鬼的随机移动增加了不确定性,考验算法的鲁棒性;
此外,游戏具有长期规划特性,agent 需要在避开鬼和收集食物之间进行权衡,胶囊机制也增加了策略的多样性。

在本次实验中,我们选择《吃豆人》作为强化学习实验的环境,探索不同算法在该环境下的表现与效果。


\subsection{基于值的强化学习方法}

% 简单介绍基于值的强化学习
强化学习方法可以大致分为基于值的方法(value-based)、基于策略的方法(policy-based)以及结合两者的混合方法(actor-critic)。
基于值的强化学习方法通过估计一个值函数(如状态值函数 $V(s)$ 或动作值函数 $Q(s,a)$)来指导智能体的决策,
是一类经典的 model-free 强化学习方法。

在本次实验中,我依次实现了蒙特卡洛学习、Q-Learning 以及 Approximate Q-Learning 三种算法。

蒙特卡洛学习与 Q-Learning 算法是相似的,两者都通过学习动作值函数 $Q(s,a)$ 来指导决策,
区别主要在于值函数的更新方式:
蒙特卡洛学习基于完整回合的实际回报进行更新,而 Q-Learning 则采用时序差分(Temporal Difference, TD)方法,每一步都可以进行增量式更新,学习效率更高。
通过实现这两种相对简单的方法,我加深了对强化学习基本原理和学习过程的理解。

然而,将 $Q(s,a)$ 直接作为学习目标,面临着状态空间爆炸的问题。
因此,我进一步实现了 Approximate Q-Learning 算法,
通过手工设计的特征函数将高维状态映射到低维特征空间,再使用线性函数逼近来估计 $Q$ 值,
从而大大降低了状态空间的复杂度,使算法能够泛化到未见过的状态,并在更复杂的环境中取得良好的表现。


\subsection{报告结构}

本文的组织结构如下:

第 2 章介绍三种强化学习方法的原理与设计,包括问题的 MDP 建模、蒙特卡洛学习、Q-Learning 以及 Approximate Q-Learning 的基本原理和算法设计;

第 3 章详细阐述各算法的具体实现,包括强化学习训练环境的搭建、各算法的代码实现细节;

第 4 章展示实验结果与分析,对比三种算法在不同地图规模下的性能表现;

第 5 章对本次实验进行总结,并提出未来可能的改进方向。