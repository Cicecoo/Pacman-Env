\section{新型储能技术}

% ============ 引言 ============
在本次课程学习中,我对新型储能技术产生了兴趣。
通过查阅文献和相关资料,我了解到储能技术是解决可再生能源波动性问题的关键手段之一。
比如风电和光伏发电受天气影响较大,发电量不稳定;另一方面像核电等发电均匀,但用户用电则集中在某些时段;
而储能系统可以在电力富余时储存能量,在电力短缺时释放,起到"削峰填谷"的作用。

根据中国能源局发布的统计数据,截至 2024 年底,全国已建成投运的新型储能项目累计装机规模达到 7376 万千瓦(73.76 GW)/1.68 亿千瓦时,
约为“十三五”末的 20 倍,较 2023 年底增长 超过 130\%,全年新增装机 4237 万千瓦(42.37 GW)/1.01 亿千瓦时。
其中,电化学储能仍占据绝对主导地位,占比超过 90\%,技术路线中以 锂离子电池为主。
通过调研,我发现储能技术虽然应用前景广阔,但在安全性、成本和市场机制等方面仍面临不少挑战。
本报告主要从电化学储能现状以及储能调控应用两个方面展开分析。


\subsection{电化学储能的发展现状与存在问题}

电化学储能是当前应用最广、发展速度最快的新型储能技术路线之一,主要包括锂离子电池、钠离子电池、全钒液流电池等类型。
该类储能技术已经在电力系统调峰调频、可再生能源并网以及分布式能源系统中得到广泛应用。

目前,锂离子电池在电化学储能市场中占据主导地位。该技术具有能量密度较高、转换效率较高、响应速度快等特点,能够满足风光发电波动调节等短时储能需求。
受动力电池产业发展的带动,锂离子储能在电池材料、制造工艺和系统集成方面持续进步,系统成本不断下降。据行业统计数据,近五年我国储能型锂电系统成本下降了近 50\%,
为大规模推广应用创造了条件。

尽管发展迅速,电化学储能仍面临一些制约产业进一步发展的问题。首先,安全性风险仍较突出。在高倍率运行和大规模集中应用的情况下,电池仍可能发生热失控等安全事故,
一些储能电站火灾案例表明电池一致性管理和系统安全设计仍需改进。其次,寿命衰减问题依然存在。锂电池在长期频繁充放电过程中容量会逐渐下降,影响系统的全生命周期经济性。
此外,锂、镍等关键原材料的资源保障风险仍然较高,价格波动可能对储能成本带来影响。

在新型技术路线方面,钠离子电池由于资源丰富、成本较低和具备一定的低温性能,逐渐成为锂电的潜在补充技术路线。全钒液流电池具有循环寿命长、安全性高等优势,
在长时储能领域具有一定应用前景,但目前仍面临能量密度较低、建设成本较高等挑战。

总体来看,我国电化学储能产业具备良好发展基础,是支撑新能源大规模开发的重要组成部分。
但要实现高质量发展,还需在安全性提升、系统集成优化、材料创新和全寿命周期成本控制等方面持续推进技术进步。


% \subsection{长时储能发展的必要性与探索方向}

% 随着风电和光伏等新能源装机容量持续增长,电力系统的调节压力逐渐加大。
% 新能源发电的日间波动和季节性差异导致电力供需矛盾在短时调节之外,还存在多日甚至多周的储能需求。
% 传统短时储能技术主要适用于分钟级至小时级调节,而应对大规模新能源消纳所需的长时储能能力仍然不足。
% 因此,发展长时储能对于实现高比例新能源并网、保障电力系统安全稳定运行具有重要意义。

% 长时储能能够实现新能源发电的时间平移,在电力过剩时储能,在需求高峰时释放,从而提高新能源利用率并减少弃风弃光现象。
% 抽水蓄能和液流电池在长时储能领域具有明显优势。
% 抽水蓄能技术成熟、效率高,但受地理条件限制,可开发容量有限。
% 液流电池储能系统具有良好的循环寿命和模块化特点,适合大规模部署,能够实现数小时至数十小时的储能。

% 近年来,我国在长时储能领域加大投入。
% 国家能源局储能发展规划提出,到2030年长时储能规模将占总储能装机约30\%,为新能源高比例消纳提供调节能力。
% 针对跨日调峰和季节性调节的技术研究也在不断推进,包括液流电池改进、压缩空气储能以及热能储存等方案。
% 这些技术通过提高能量密度、降低成本以及延长寿命,为新能源大规模消纳提供支持。

% 长时储能仍面临技术成本高、系统效率和寿命需优化以及调度策略复杂等挑战。
% 大规模长时储能需要合理设计充放电周期,保证系统与电网协同运行,实现新能源与负荷的高效匹配。


\subsection{储能调控体系建设与应用需求分析}

储能在新能源电力系统中的作用不仅体现在容量建设层面,还涉及系统运行调控能力的提升。
随着新能源装机规模不断扩大,其出力波动性和不确定性逐步增加,电网对储能的调控和管理提出了更高要求。
因此,需要从系统规划、运行调度和管理机制等方面构建完善的储能调控体系。

储能系统的并网调度是保障电力系统安全稳定运行的重要环节。储能可以参与峰谷调节、频率调节和备用容量提供等多项辅助服务。
在短时调节中,储能具备快速响应特性,可实现秒级至分钟级的功率调节,提升系统频率稳定性。在长时调节中,储能需要结合新能源发电预测和负荷预测进行优化运行,
通过合理安排充放电策略,提高电力供需平衡能力。同时,储能应与传统调峰电源进行协调运行,合理制定调度策略,有助于提升系统整体运行效率。

合理的容量配置与空间布局是储能发挥作用的基础。在新能源占比较高、输电能力受限的地区,分布式储能能够有效缓解局部电网压力,提高新能源就地消纳水平。
在主干电网节点或重要负荷中心布局集中式储能,则可承担跨区调峰、应急备用和电力支撑等任务,有助于提升电网运行的安全性与灵活性。

同时,储能调控体系的建设还需要配套完善的运行管理机制。在系统运行过程中,应加强设备状态监测、电池健康评估和充放电周期管理,以延长系统寿命并保障运行安全。
此外,还需建立储能参与电力市场的价格机制,使其能够在调频市场、辅助服务市场和容量市场中获得合理收益,从而提升投资积极性,推动储能产业可持续发展。


\subsection{新型储能面临的主要问题}

虽然新型储能在支撑新能源发展和构建新型电力系统中具有重要作用,但在推广应用过程中仍面临多方面挑战。
其中,成本问题仍是制约其大规模发展的关键因素之一。当前电化学储能和长时储能系统的初始投资仍较高,虽然近几年成本有所下降,但整体仍高于传统调峰电源,投资回收周期较长,
经济性有待进一步提升。

安全性问题依然突出。部分储能系统仍存在热失控、过充过放、电解液泄漏等潜在风险,特别是在大型集中式储能电站中,一旦发生事故可能导致设备损坏甚至造成人员伤亡。
因此,储能系统的安全设计与运行保障仍需加强。

寿命衰减也是影响储能可靠性的重要因素。不同储能技术的循环寿命和性能保持能力存在差异。例如,液流电池具有较长寿命但能量密度较低;
而锂离子电池虽然能量密度较高,但循环寿命和容量保持率受运行工况影响较大。

此外,市场机制和政策体系尚需完善。目前储能参与电力市场的机制仍不健全,在辅助服务市场、电力现货市场和容量补偿机制方面仍存在政策缺口,
导致储能收益模式不清晰,投资积极性不足。同时,在电网规划、容量管理和跨区域调度等方面缺乏统一的标准和管理规范,制约行业健康发展。

\subsection{发展对策与建议}

为推动新型储能的健康发展,需要从技术进步、工程应用和政策机制三个方面协同推进。在技术方面,应提升电化学储能系统的安全性和寿命性能,同时推动新型储能技术的研发和示范应用,
提高长时储能技术的工程可行性和经济性。在工程应用方面,应加强储能与新能源电站、电网调控和用户侧需求的协同部署,推动建设多类型示范项目,探索适合不同应用场景的配置模式和运行策略。
在政策和市场层面,应完善储能参与电力市场的机制,明确储能在辅助服务、电力调度中的功能定位,建立合理的收益机制和运行规范,增强投资可预期性,促进产业持续发展。