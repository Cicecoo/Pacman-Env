\section{总结}

从课堂、教材和网络资料中可以理解强化学习算法的数学原理和理论流程,但真正去实现代码、调试训练过程时仍会遇到许多问题。
而在修改代码的过程中,对于状态表示、动作选择、奖励设计等核心概念也会有新的认识。

实验中在环境搭建和特征设计部分花费的时间最长,主要是最初并不理解强化学习环境的接口规范,也不清楚状态空间和动作空间应该如何定义。
通过查找 Gymnasium 文档了解标准接口的要求;逐步实现状态从原始游戏数据到特征向量的转换过程,理解智能体如何感知环境;
对照各算法的更新公式,逐渐摸清 Q 值表、特征权重等结构的作用,之后才顺利完成算法实现。

在实现表格式方法(MC Learning、Q-Learning)时发现,即使是小规模的 7×7 地图,状态空间已经非常庞大,训练时很难遍历所有状态。
这让我真正体会到"维度灾难"不仅是理论上的概念,在实际应用中会严重制约算法的可行性。
而在引入函数近似、并通过手工设计特征将高维状态压缩到低维空间后,智能体仅用 1000 个 episode 就学会了有效的策略,并且有很强的泛化能力。
这让我深刻认识到算法和特征设计的重要性。

在特征设计过程中收获颇丰,通过分析权重演化发现智能体会优先学习危险回避,然后才学会利用恐慌鬼获取高分,这种分阶段学习的现象很有意思。
但遗憾的是没有足够的时间去尝试更多的算法变体,在调试代码细节上花了比较多的时间,接触新方法较少。
此外,当前特征主要基于简单的距离计算,对于路径规划、区域控制等更高层次的策略缺乏建模,导致在复杂地图上的表现还不够理想。

未来还可以
尝试深度强化学习方法(如 DQN),用神经网络自动学习特征,避免手工设计的局限性;
以及引入更复杂的训练技巧,如经验回放、目标网络等,提高样本利用效率;

总体而言,本次实验让我对强化学习有了更深入的理解,不仅停留在理论层面,而是通过实践体会到了算法设计、环境建模、特征工程的重要性和挑战性。